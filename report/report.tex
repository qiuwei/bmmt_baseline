\documentclass[a4paper]{article}

\usepackage[english]{babel}
\usepackage[utf8]{inputenc}
\usepackage{amsmath}
\usepackage{graphicx}
\usepackage[colorinlistoftodos]{todonotes}

\title{A Short Report of My Solution to the Assignment}

\author{Wei Qiu}

\date{\today}

\begin{document}
\maketitle

\section{Directory structure}
I reorganized the files to make my solution easier to explain.
\begin{itemize}
    \item All the data is located at data.
    \item Test data is in data/test. And test.en is copied to reference0.
    \item Train data is in data/train. 
    \item Dev data is in data/dev.
\end{itemize}

\section{Problem 1 \& 3}

A baseline system is not so interesting.
I address subproblem 1 and 3 together, i.e, I get a fine-tuned system using the development set. 
The config file is located at baseline/config.

\section{Problem 2}
The first idea comes to my mind is to simply append all of the terminology table to the training table. The sript split_term.py is used to split the terminology into seperate files. Then I manually append them to the corresponding training files.

The trained model using this idea is located at prob2. The correspoding training data is at data/train\_term.

A further research shows that moses has an advanced feature that can let the user specify how to translate certain words by marking up them during preprocessing.
This may be the best approach to deal with external teminnology dictionary.




\section{Problem 5: further improvement}
There are at least two ways to improve the translation quality:
\begin{itemize}
    \item As German is an agglutinative language, splitting up the compound words can reduce the sparsity. 
    \item As the word order in German has much more freedom than English, a preprocessing step which always normalize German sentence to SVO can reduce the sparisity as well.

\end{itemize}
\end{document}
